\section{Textursynthese}

Die Textursynthese ist ein alternativer Ansatz für die Erstellung von Texturen, der es erlaubt auch ohne spezielle Vorkenntnisse qualitativ hochwertige Ergebnisse zu erzielen.
Der Benutzer muss lediglich ein Beispielmuster und eventuell benötigte Konfigurationsparameter an die Textursynthese übergeben, die aus diesen Daten eine Textur synthetisiert.
Die resultierende Textur zeichnet sich dadurch aus, dass sie eine beliebige Größe annehmen kann, aber weiterhin sichtbare Ähnlichkeiten zu dem Beispielmuster aufweist.
Die Textursynthese kümmert sich dabei automatisiert und im Idealfall ohne aufwändige Benutzereingaben um die Schwierigkeiten bei der Erstellung von Texturen.
Sie berücksichtigt die visuellen Charakteristiken des Beispielmusters und vermeidet dabei zeitgleich auffällige Wiederholungen oder Unnatürlichkeiten in der synthetisierten Textur.
